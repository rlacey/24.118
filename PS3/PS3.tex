%%%%%%%%%%%%%%%%%%%%%%%%%%%%%%%%%%%%%%%%%
% University/School Laboratory Report
% LaTeX Template
% Version 3.0 (4/2/13)
%
% This template has been downloaded from:
% http://www.LaTeXTemplates.com
%
% Original author:
% Linux and Unix Users Group at Virginia Tech Wiki 
% (https://vtluug.org/wiki/Example_LaTeX_chem_lab_report)
%
% License:
% CC BY-NC-SA 3.0 (http://creativecommons.org/licenses/by-nc-sa/3.0/)
%
%%%%%%%%%%%%%%%%%%%%%%%%%%%%%%%%%%%%%%%%%

%----------------------------------------------------------------------------------------
%	PACKAGES AND DOCUMENT CONFIGURATIONS
%----------------------------------------------------------------------------------------

\documentclass{article}

\usepackage[version=3]{mhchem} % Package for chemical equation typesetting
\usepackage{siunitx} % Provides the \SI{}{} command for typesetting SI units

\usepackage{graphicx}
\usepackage{caption}
\usepackage{subcaption}

\usepackage{float}

\usepackage[T1]{fontenc} % allow small bold caps

\setlength\parindent{0pt} % Removes all indentation from paragraphs

\renewcommand{\labelenumi}{\alph{enumi}.} % Make numbering in the enumerate environment by letter rather than number (e.g. section 6)

\usepackage[margin=1in]{geometry}

\usepackage{amssymb}

%\usepackage{times} % Uncomment to use the Times New Roman font

%----------------------------------------------------------------------------------------
%	Title
%----------------------------------------------------------------------------------------

\begin{document}
\pagenumbering{gobble}

\title{24.118: Paradox and Infinity}
\author{
  Ryan Lacey <rlacey@mit.edu>\\
  \footnotesize \texttt{Collaborator(s): Evan Thomas, Rodrigo Paniza}
}
        
\maketitle
        


\begin{enumerate}
\item[1.]
	\begin{enumerate}
	\item[(a)] 
			$P(\text{One red ball}) = stuff$
	\item[(b)] 
			$P(\text{Exactly two heads}) = \dfrac{\text{number of ways to get two heads}}{\text{number of possible outcomes of five flips}} = \dfrac{\binom{5}{2}}{2^5} = \dfrac{10}{32}$
	\item[(c)] 
			The outcomes are equally likely. A specific ordering of ten coin flips. regardless of the flip results, has probability $\frac{1}{2^{10}}$ of appearing.
	\item[(d)] 
			$P(\text{No aces})= \dfrac{\text{number of ways to get no aces}}{\text{number of possible card hands}} = \dfrac{\binom{48}{5}}{\binom{52}{5}} = \dfrac{35673}{54145} \approx 0.658$\\
			
			$P(\text{One ace})= \dfrac{\text{number of oways to get one ace}}{\text{number of possible card hands}} = \dfrac{\binom{4}{1}\binom{48}{4}}{\binom{52}{5}} = \dfrac{3243}{10829} \approx 0.299$\\
			
			$P(\text{At least two aces})= 1 - \left(P(\text{No aces}) + P(\text{One ace})\right) = 1 - \left(\dfrac{35673}{54145} + \dfrac{\binom{4}{1}\binom{48}{4}}{\binom{2257}{54145}} \right) \approx 0.041$
	\end{enumerate}


\bigskip


\end{enumerate}

\end{document}