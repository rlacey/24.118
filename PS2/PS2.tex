%%%%%%%%%%%%%%%%%%%%%%%%%%%%%%%%%%%%%%%%%
% University/School Laboratory Report
% LaTeX Template
% Version 3.0 (4/2/13)
%
% This template has been downloaded from:
% http://www.LaTeXTemplates.com
%
% Original author:
% Linux and Unix Users Group at Virginia Tech Wiki 
% (https://vtluug.org/wiki/Example_LaTeX_chem_lab_report)
%
% License:
% CC BY-NC-SA 3.0 (http://creativecommons.org/licenses/by-nc-sa/3.0/)
%
%%%%%%%%%%%%%%%%%%%%%%%%%%%%%%%%%%%%%%%%%

%----------------------------------------------------------------------------------------
%	PACKAGES AND DOCUMENT CONFIGURATIONS
%----------------------------------------------------------------------------------------

\documentclass{article}

\usepackage[version=3]{mhchem} % Package for chemical equation typesetting
\usepackage{siunitx} % Provides the \SI{}{} command for typesetting SI units

\usepackage{graphicx}
\usepackage{caption}
\usepackage{subcaption}

\usepackage{float}

\usepackage[T1]{fontenc} % allow small bold caps

\setlength\parindent{0pt} % Removes all indentation from paragraphs

\renewcommand{\labelenumi}{\alph{enumi}.} % Make numbering in the enumerate environment by letter rather than number (e.g. section 6)

\usepackage[margin=1in]{geometry}

\usepackage{amssymb}

%\usepackage{times} % Uncomment to use the Times New Roman font

%----------------------------------------------------------------------------------------
%	Title
%----------------------------------------------------------------------------------------

\begin{document}
\pagenumbering{gobble}

\title{24.118: Paradox and Infinity}
\author{
  Ryan Lacey <rlacey@mit.edu>\\
  \footnotesize \texttt{Collaborator(s): Evan Thomas, Rodrigo Paniza}
}
        
\maketitle
        


\begin{enumerate}
\item[1.] The expected value of partying is greater than the expected value of studying, therefore you should choose to party!     
               \begin{flalign*}
               E(party) &= V(party, hard)P(hard | party) + V(party,easy)P(easy | party)&\\
                            &= (-25)(0.2) + (35)(0.8)\\
                            &= 23\\
               \end{flalign*}
               
               \begin{flalign*}
               E(study) &= V(study, hard)P(hard | study) + V(study, easy)P(study, easy)&\\
                            &= (18)(0.2) + (18)(0.8)\\
                            &= 18\\               
               \end{flalign*}            
                            
\bigskip

\item[2.] The expected value of studying is greater than the expected value of partying, therefore you should choose to study...    
               \begin{flalign*}
               E(party) &= V(party, hard)P(hard | party) + V(party,easy)P(easy | party)&\\
                            &= (-25)(0.7) + (35)(0.3)\\
                            &= -7\\
               \end{flalign*}
               
               \begin{flalign*}
               E(study) &= V(study, hard)P(hard | study) + V(study, easy)P(study, easy)&\\
                            &= (18)(0.2) + (18)(0.8)\\
                            &= 18\\               
               \end{flalign*}   
                                                
\newpage

\item[3.] You can party or study with equal consequence if the professor gives exams such that:\\

              $P(hard | party) = \frac{17}{60}$ and $P(hard | study) = \frac{1}{5}$\\
              
               We know from \texttt{(2)} that expected value of partying when $P(hard | study) = \frac{1}{5}$ is 18. So solving for the conditional exam probability for the same expected value gives us what we want.\\
               \begin{flalign*}
               18 &= (-25)(P) + (25)(1-P)&\\
               -17 &=-60P\\
                P &= \frac{17}{60}\\               
               \end{flalign*}     
\bigskip

\item[4.]   According to Evidential Decision Theory you should TwoBox. As long as the probablility of Closed having \$1M is nonzero, then the expected value of TwoBox must be greater than OneBox because the \$100 is guaranteed for both situations. Slightly more formally:\\
                         
$E(OneBox) = P(open = \$100)V(open) = (1)(100) = 100$\\
                         
$E(TwoBox) = P(open = \$100)V(open) + P(closed = \$0)V(closed)  + P(closed = \$1M)V(closed)  = 100 + \alpha \$1M$ for some $\alpha > 0$

\bigskip

\item[5.] Even under the predictor condition Evidential Decision Theory supports the OneBox strategy. In fact the OneBox strategy will be favored until the predictor's correctness approaches 50 \%, in which case the expected value of TwoBox will become  greater. 
               \begin{flalign*}
               E(OneBox) &= V(full)P(full|OneBox) + V(empty)P(empty|OneBox)&\\
                            &= (1000000)(0.8) + (0)(0.2)\\
                            &= 800,000\\
               \end{flalign*}
               
               \begin{flalign*}
               E(TwoBox) &= V(full)P(full|TwoBox) + V(empty)P(empty|TwoBox)&\\
                            &= (1000010)(0.2) + (10)(0.8)\\
                            &= 200,010\\               
               \end{flalign*}   

\item[6.] I am an advocate for OneBox. In the standard conditions for tis problem, with a high accuracy predictor and large difference in box values, choosing OneBox has the higher expected value. Even if the starting conditions were modified we could find the expected value and would choose OneBox until TwoBox had greater value. Effectively the choice you make in how many boxes to take dictates how much money was put onto the closed box in the past. To accept this we must abandon the notion of causality in the sense of past events leading to future events exclusively to instead consider that this relation can occur both forwards and backwards through time. While not physically possible, it is a world that is logically possible. Consider the extreme in which the predictor is 100\% accurate. Then TwoBox guarantees that no money was put into the closed box in the past. \\     

\bigskip

\item[7.] You should rat out Jones, because we can assume that he will rat you out to minimize his own time in jail given that he acts fully rationally and that value only stems from time spent in jail. This is rational because not ratting the other person out will result in either 10 or 10,000 days in jail while going through with ratting the other out will result in either 0 or 9,000 days in jail. Clearly the potential sentence lengths for ratting the other out are less than those of remaining silent.

\bigskip

\item[8.] To rephrase this question, I consider what I would do if I knew how the other prisoner thinks. To this I answer that I would remain silent knowing that Jones (the other me) would also remain silent. This maximizes the expected value for the group rather than just for the individual. Note that while I maintain that value only derives from prison time served (no guilt, reputation, etc), I have changed how the individual calculates this value to incorporate both parties instead of just oneself.

\bigskip

\item[9.] The conditions of this are unclear as to whether this is adversarial Jones or clone of me Jones. Adversarial Jones that only cares about himself would rat you out each and every time because that minimizes his expected time in jail. Knowing that Jones is a jerk, confirmed after each day by being told his decision, I would also choose to rat him out in order to minimize and counteract the actions he has taken. If Jones was clone me, however, then we would happily wait in jail silent for ten sentences and receive confirmation that we are good people after the end of each day. The total amount of time we serve if we do this (10 days per sentence if silent)*(10 crimes sentenced) = 100 days served is less than the time served if we ratted each other out even once, which causes a 9,000 day sentence for both.

\end{enumerate}

\end{document}