%%%%%%%%%%%%%%%%%%%%%%%%%%%%%%%%%%%%%%%%%
% University/School Laboratory Report
% LaTeX Template
% Version 3.0 (4/2/13)
%
% This template has been downloaded from:
% http://www.LaTeXTemplates.com
%
% Original author:
% Linux and Unix Users Group at Virginia Tech Wiki 
% (https://vtluug.org/wiki/Example_LaTeX_chem_lab_report)
%
% License:
% CC BY-NC-SA 3.0 (http://creativecommons.org/licenses/by-nc-sa/3.0/)
%
%%%%%%%%%%%%%%%%%%%%%%%%%%%%%%%%%%%%%%%%%

%----------------------------------------------------------------------------------------
%	PACKAGES AND DOCUMENT CONFIGURATIONS
%----------------------------------------------------------------------------------------

\documentclass{article}

\usepackage[version=3]{mhchem} % Package for chemical equation typesetting
\usepackage{siunitx} % Provides the \SI{}{} command for typesetting SI units

\usepackage{graphicx}
\usepackage{caption}
\usepackage{subcaption}

\usepackage{float}

\usepackage[T1]{fontenc} % allow small bold caps

\setlength\parindent{0pt} % Removes all indentation from paragraphs

\renewcommand{\labelenumi}{\alph{enumi}.} % Make numbering in the enumerate environment by letter rather than number (e.g. section 6)

\usepackage[margin=1in]{geometry}

\usepackage{amssymb}

%\usepackage{times} % Uncomment to use the Times New Roman font

%----------------------------------------------------------------------------------------
%	Title
%----------------------------------------------------------------------------------------

\begin{document}
\pagenumbering{gobble}

\title{24.118: Paradox and Infinity}
\author{
  Ryan Lacey <rlacey@mit.edu>\\
  \footnotesize \texttt{Collaborator(s): Jorge Perez, Evan Thomas, Alvin Jeon}
}
        
\maketitle
        


\begin{enumerate}
\item[1.] The expected value of partying is greater than the expected value of studying, therefore you should choose to party!     
               \begin{flalign*}
               E(party) &= V(party, hard)P(hard | party) + V(party,easy)P(easy | party)&\\
                            &= (-25)(0.2) + (35)(0.8)\\
                            &= 23\\
               \end{flalign*}
               
               \begin{flalign*}
               E(study) &= V(study, hard)P(hard | study) + V(study, easy)P(study, easy)&\\
                            &= (18)(0.2) + (18)(0.8)\\
                            &= 18\\               
               \end{flalign*}            
                            
\bigskip

\item[2.] The expected value of studying is greater than the expected value of partying, therefore you should choose to study...    
               \begin{flalign*}
               E(party) &= V(party, hard)P(hard | party) + V(party,easy)P(easy | party)&\\
                            &= (-25)(0.7) + (35)(0.3)\\
                            &= -7\\
               \end{flalign*}
               
               \begin{flalign*}
               E(study) &= V(study, hard)P(hard | study) + V(study, easy)P(study, easy)&\\
                            &= (18)(0.2) + (18)(0.8)\\
                            &= 18\\               
               \end{flalign*}   
                                                
\newpage

\item[3.] You can party or study with equal consequence if the professor gives exams such that:\\

              $P(hard | party) = \frac{17}{60}$ and $P(hard | study) = \frac{1}{5}$\\
              
               We know from \texttt{(2)} that expected value of partying when $P(hard | study) = \frac{1}{5}$ is 18. So solving for the conditional exam probability for the same expected value gives us what we want.\\
               \begin{flalign*}
               18 &= (-25)(P) + (25)(1-P)&\\
               -17 &=-60P\\
                P &= \frac{17}{60}\\               
               \end{flalign*}     
\bigskip

\item[4.]   According to Evidential Decision Theory you should TwoBox. As long as the probablility of Closed having \$1M is nonzero, then the expected value of TwoBox must be greater than OneBox because the \$100 is guaranteed for both situations. Slightly more formally:\\
                         
$E(OneBox) = P(open = \$100)V(open) = (1)(100) = 100$\\
                         
$E(TwoBox) = P(open = \$100)V(open) + P(closed = \$0)V(closed)  + P(closed = \$1M)V(closed)  = 100 + \alpha \$1M$ for some $\alpha > 0$

\bigskip

\item[5.] Even under the predictor condition Evidential Decision Theory supports the OneBox strategy. In fact the OneBox strategy will be favored until the predictor's correctness approaches 50 \%, in which case the expected value of TwoBox will become  greater. 
               \begin{flalign*}
               E(OneBox) &= V(full)P(full|OneBox) + V(empty)P(empty|OneBox)&\\
                            &= (1000000)(0.8) + (0)(0.2)\\
                            &= 800,000\\
               \end{flalign*}
               
               \begin{flalign*}
               E(TwoBox) &= V(full)P(full|TwoBox) + V(empty)P(empty|TwoBox)&\\
                            &= (1000010)(0.2) + (10)(0.8)\\
                            &= 200,010\\               
               \end{flalign*}   
               
\end{enumerate}

\end{document}