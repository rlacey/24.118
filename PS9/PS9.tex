%%%%%%%%%%%%%%%%%%%%%%%%%%%%%%%%%%%%%%%%%
% University/School Laboratory Report
% LaTeX Template
% Version 3.0 (4/2/13)
%
% This template has been downloaded from:
% http://www.LaTeXTemplates.com
%
% Original author:
% Linux and Unix Users Group at Virginia Tech Wiki 
% (https://vtluug.org/wiki/Example_LaTeX_chem_lab_report)
%
% License:
% CC BY-NC-SA 3.0 (http://creativecommons.org/licenses/by-nc-sa/3.0/)
%
%%%%%%%%%%%%%%%%%%%%%%%%%%%%%%%%%%%%%%%%%

%----------------------------------------------------------------------------------------
%	PACKAGES AND DOCUMENT CONFIGURATIONS
%----------------------------------------------------------------------------------------

\documentclass{article}

\usepackage[version=3]{mhchem} % Package for chemical equation typesetting
\usepackage{siunitx} % Provides the \SI{}{} command for typesetting SI units

\usepackage{graphicx}
\usepackage{caption}
\usepackage{subcaption}

\usepackage{float}

\usepackage[T1]{fontenc} % allow small bold caps

\setlength\parindent{0pt} % Removes all indentation from paragraphs

\renewcommand{\labelenumi}{\alph{enumi}.} % Make numbering in the enumerate environment by letter rather than number (e.g. section 6)

\usepackage[margin=1in]{geometry}

\usepackage{amssymb}

%\usepackage{times} % Uncomment to use the Times New Roman font

%----------------------------------------------------------------------------------------
%	Title
%----------------------------------------------------------------------------------------

\begin{document}
\pagenumbering{gobble}

\title{24.118: Paradox and Infinity}
\author{
  Ryan Lacey <rlacey@mit.edu>\\
  \footnotesize \texttt{Collaborator(s): Evan Thomas}
}
        
\maketitle
        


\begin{enumerate}
\item[1.]
	\begin{enumerate}
		\item[(a)]
			$\forall a .\forall b .\forall c. \; \neg ((a * (a * a)) + (b * (b * b)) = (c * (c * c)))$\\
		\item[(b)]
			For any $x$ there exists a number $y$ that is greater in value than $x$ and that is only divisible by one and itself (a prime number).\\
			
			Define there exists -- $\exists x$: $\neg \forall x \neg \Phi$\\
			
			Define greater than -- $x_2>x_1$: $\exists y. \; ((x_1 + y = x_2) \land \neg (y = 0))$\\
			
			Define prime -- $Prime(x)$: $\forall y_1. \forall y_2. \; \neg ((y_1 * y_2 = x) \land (y_1 > 1) \land (y_2 > 1))$\\
			
			$\forall x. \exists y. \; ((y > x)  \land Prime(y))$\\
	\end{enumerate}
\item[2.]
	\begin{enumerate}
		\item[(a)]
			The formula $Pair(n,a,b)$ is expressed in $\mathcal{L}$ as  $n = 2^{(a+1)} * 3^{(b+1)}$\\
		\item[(b)]
			The formula $Inc(n,m)$ is expressed in $\mathcal{L}$ as  $\exists p. \exists c. \; ((n = p^{(m+1)}  * c) \land (Prime(p) ) \land (c>0))$\\
		\item[(c)]
			Define implies -- $x \implies y$: $\neg (x \land \neg y)$\\
		
			The formula $NSeq(n,k)$ is expressed in $\mathcal{L}$ as the conjunction of the following:\\
			
			All elements are pairs\\
			$\forall x. \; Inc(n,x) \implies \exists u. \exists v. \; Pair(x, u, v)$\\
			
			Each pair is unique\\
			$\forall x.\forall y. \; (Inc(n,x) \land Inc(n,y) \implies \exists z. \exists z_1. \exists z_2. \; \neg (Pair(x, z, z_1) = Pair(y, z, z_2)))$\\
			
			Pairs up to $k$ included\\
			$\forall c. \; (((k+1>c) \land (c>0)) \implies \exists x.\exists y. \; (Inc(n,x) \land (Pair(x, c, y))))$\\
			
		\item[(d)]
			The formula $Seq(n,k,i,m)$ is expressed in $\mathcal{L}$ as  $\exists x. \; (NSeq(n,k) \land (x = Pair(i,m)) \land (Inc(n,x))$\\
	\end{enumerate}
\end{enumerate}
\end{document}