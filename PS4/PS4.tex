%%%%%%%%%%%%%%%%%%%%%%%%%%%%%%%%%%%%%%%%%
% University/School Laboratory Report
% LaTeX Template
% Version 3.0 (4/2/13)
%
% This template has been downloaded from:
% http://www.LaTeXTemplates.com
%
% Original author:
% Linux and Unix Users Group at Virginia Tech Wiki 
% (https://vtluug.org/wiki/Example_LaTeX_chem_lab_report)
%
% License:
% CC BY-NC-SA 3.0 (http://creativecommons.org/licenses/by-nc-sa/3.0/)
%
%%%%%%%%%%%%%%%%%%%%%%%%%%%%%%%%%%%%%%%%%

%----------------------------------------------------------------------------------------
%	PACKAGES AND DOCUMENT CONFIGURATIONS
%----------------------------------------------------------------------------------------

\documentclass{article}

\usepackage[version=3]{mhchem} % Package for chemical equation typesetting
\usepackage{siunitx} % Provides the \SI{}{} command for typesetting SI units

\usepackage{graphicx}
\usepackage{caption}
\usepackage{subcaption}

\usepackage{float}

\usepackage[T1]{fontenc} % allow small bold caps

\setlength\parindent{0pt} % Removes all indentation from paragraphs

\renewcommand{\labelenumi}{\alph{enumi}.} % Make numbering in the enumerate environment by letter rather than number (e.g. section 6)

\usepackage[margin=1in]{geometry}

\usepackage{amssymb}

%\usepackage{times} % Uncomment to use the Times New Roman font

%----------------------------------------------------------------------------------------
%	Title
%----------------------------------------------------------------------------------------

\begin{document}
\pagenumbering{gobble}

\title{24.118: Paradox and Infinity}
\author{
  Ryan Lacey <rlacey@mit.edu>\\
  \footnotesize \texttt{Collaborator(s): Evan Thomas}
}
        
\maketitle
        


\begin{enumerate}
\item[1.]
	\begin{enumerate}
	\item[(a)]
		This randomized procedure is the same as the one listed in the lecture notes, from which we draw the fact that selecting a number in the range of [0,$k$]  is $k$. This is because we have a uniform distribution of probabilities for any infinite series under the given method. We desire the probability of a value being in the range [$m/2^k$, $n/2^k$]. Another way to state this is the probability of getting a value in the range [0, $n/2^k$], excluding the values in the range [0, $m/2^k$]. The probability of a value in [0, $n/2^k$] is $n/2^k$ from what was stated previously. Excluding the values in [0, $m/2^k$] means we are simply removing $m/2^k$ from that probability, which we can just subtract off due to the uniform distribution. Therefore the probability of a value in [$m/2^k$, $n/2^k$] is $n/2^k - m/2^k = \frac{n-m}{2^k}$.\\
	\item[(b)]
		To have a $\frac{1}{3}$ probability of being above one half we need to simulate a fair three-sided coin. To achieve this we flip two fair coins once. If the coins land \texttt{TT} then we flip them again. Else if the coins land \texttt{HH} then we assign the beginning of our sequence value \texttt{0.1}. Else the coins landed \texttt{TH} or \texttt{HT} and we assign the beginning of our sequence \texttt{0.0}. After this initialization the remaining bits of the sequence are set in the standard manner of 1 for heads and 0 for tails.\\
		
		All of the coin combinations have equal probability. Throwing out the possibility for \texttt{TT}, then each of the other three combinations has probability $\frac{1}{3}$. Setting \texttt{0.1} to head the sequence for \texttt{HH} means that with probability $\frac{1}{3}$ we ensure that the output will have value greater than ore equal to one half. Intuitively the remaining options make it so the output is less than one half with probability $\frac{1}{3} + \frac{1}{3} = \frac{2}{3}$. 
	\end{enumerate}

\newpage

\item[2.]
	\begin{enumerate}
	\item[(a)] 
		Define a binary string constructed from the coin tosses as described in problem \texttt{(1a)}, namely that a head corresponds to a  \texttt{0} and a tail corresponds to a  \texttt{1}. The string is prepended with \texttt{0.} so that the number constructed is a real positive number up to value one. The side-length output of the cube machine is the result of the infinite coin flips converted into a value [0,1].
	\item[(b)] 
		Flip a fair coin once. If its heads, then assign \text{0.111} to be the beginning of the sequence. Set the bits from the fourth position onward in the standard manner of 1 for heads and 0 for tails. If the first flip was tails, then assign \text{0.0} to be the beginning of the sequence. Set the bits from the second position onward in the standard manner. The value of \text{0.111} is $\frac{1}{2} + \frac{1}{4} + \frac{1}{8} = \frac{7}{8}$. Thus with probability $\frac{1}{2}$ we are setting our output to be above or below $\frac{7}{8}$ from the first flip. All subsequent flips merely determine where in the range from $[0,\frac{7}{8}]$ or $[\frac{7}{8},1]$ the side length will be.
	\end{enumerate}

\bigskip

\item[3.]
	\begin{enumerate}
	\item[(a)]
		\begin{enumerate}
		\item[(i)]
			$Reflexive$ - It is obvious that $f_1$ is in the same orbit as $f_1$. There is a finite number of differences between them, namely zero differences, as they are one and the same.
		\item[(ii)]
			$Symmetric$ - If $f_1$ is in the same orbit as $f_2$ then they must differ at a finite number of locations. Let us say that the indicies where $f_1$ differs from $f_2$ is represented in a list $L$. We now want to consider the locations in which $f_2$ differs from $f_1$. The list that represents these differences is also $L$ and must be of finite length due to how we defined $L$ initially.
		\item[(i)]
			$Transitive$ - Let us extend the terminology of \texttt{(ii)} to define three lists $L_{12}$, $L_{23}$, and $L_{13}$. The subscripts of each list denote which $f$'s it holds the indexed differences of. Assume that each $f$ is distinct, else we wold have one of the previous two cases and the respective $L$ would be an empty-set. Each $L$ is of finite size due to the nature of its construction. Therefore the union of the lists results in a list that must also be of finite size. At most the size of $L_a \bigcup L_b$ is $|L_a| + |L_b|$. Each $f$ has a finite number of differences from each other $f$, so must all be in the same orbit.
		\end{enumerate}
	\item[(b)]
		Since $f_0$ consists of an infinite string of 0s, then everything else in its orbit differs by have a finite amount of bits flipped, ie. a finite amounts of 1s. Each function $f_i$ outputs the natural number $i$, represented in binary notation. For example, if $i=9$ then the output of $f_9$ has binary representation $1001$ and would differ from $f_0$ in two locations, indexed (1,4). As each natural number has a unique representation in binary, each $f_i$ is a unique mapping of the natural numbers.
	\item[(c)]
		We take a similar approach to \texttt{(3b)}. Arbitrarily pick a function $f$ of $O$ to be $f_0$ (Axiom of Choice). All other functions $f_i$ output a natural number $i$ represented in binary. At each location in which the output has a 1 we flip the bit of $f_0$. The result has a finite number of differences of $f_0$, just the bits that were flipped, and therefore must be a member of $O$. In this manner we can get a mapping to all natural numbers.
	\end{enumerate}
\end{enumerate}

\end{document}