%%%%%%%%%%%%%%%%%%%%%%%%%%%%%%%%%%%%%%%%%
% University/School Laboratory Report
% LaTeX Template
% Version 3.0 (4/2/13)
%
% This template has been downloaded from:
% http://www.LaTeXTemplates.com
%
% Original author:
% Linux and Unix Users Group at Virginia Tech Wiki 
% (https://vtluug.org/wiki/Example_LaTeX_chem_lab_report)
%
% License:
% CC BY-NC-SA 3.0 (http://creativecommons.org/licenses/by-nc-sa/3.0/)
%
%%%%%%%%%%%%%%%%%%%%%%%%%%%%%%%%%%%%%%%%%

%----------------------------------------------------------------------------------------
%	PACKAGES AND DOCUMENT CONFIGURATIONS
%----------------------------------------------------------------------------------------

\documentclass{article}

\usepackage[version=3]{mhchem} % Package for chemical equation typesetting
\usepackage{siunitx} % Provides the \SI{}{} command for typesetting SI units

\usepackage{graphicx}
\usepackage{caption}
\usepackage{subcaption}

\usepackage{float}

\usepackage[T1]{fontenc} % allow small bold caps

\setlength\parindent{0pt} % Removes all indentation from paragraphs

\renewcommand{\labelenumi}{\alph{enumi}.} % Make numbering in the enumerate environment by letter rather than number (e.g. section 6)

\usepackage[margin=1in]{geometry}

\usepackage{amssymb}

%\usepackage{times} % Uncomment to use the Times New Roman font

%----------------------------------------------------------------------------------------
%	Title
%----------------------------------------------------------------------------------------

\begin{document}
\pagenumbering{gobble}

\title{24.118: Paradox and Infinity}
\author{
  Ryan Lacey <rlacey@mit.edu>\\
  \footnotesize \texttt{Collaborator(s): Jorge Perez, Evan Thomas, Alvin Jeon}
}
        
\maketitle
        


\begin{enumerate}
\item[1.]
	Yes, time travel is compatible with determinism. Any acts of the traveler throughout his travels can be predicted because with determinism the state of the universe is known at any point. Therefore if the traveler goes to the past and interacts with events there it is because before the traveler ever went into his time machine there was a person that did those past events (that person being the traveler himself). In essence the actions of the traveler in the present (going in a time machine to the past) are his fulfillment of the actions that had already happened. 

\bigskip

\item[2.]
	Yes, determinism is compatible with free will. This compatibility requires us consider small miracles, which we define as small localized changes of the brain. Through determinism we state that we can know the state of the world at any time. Say that from the beginning of the universe we have $D_1$ defined to be the determinable state at any time. At the point our small miracle occurs, however, $D_1$ is no longer valid and we assume a new world which can be predicted at any time by some $D-2$. Except for the point at which the miracle occurs determinism is maintained. However because the world is different, there is another deterministic set that describes it. Thus, taking the small miracle to be a logical possibility, a subject could have made a different decision while maintaining determinism before and after the decision was made, so the subject has free will.

\bigskip

\item[3.]
	No, Bruno cannot succeed. For Bruno to succeed we would have not only a physical impossibility, but also a logical inconsistency. Let us assume the opposite, that Bruno did go back in time and kill his grandfather. The grandfather then has no children before death and Bruno could not have been born. Events in the past (the grandfather's death) prevent the future event (Bruno going back in time to kill his grandfather). We maintain a world with only one timeline, ie. it is not the case that we have some $W_1$ in which Bruno exists to go back in time and some $W_2$ in which the grandfather was killed so Bruno was never born. With only one world it is logically inconsistent for both events grandfather killed and Bruno going back in time to exist. A reason Bruno failed could have been that his gun jammed, so he gave up.

\bigskip

\item[4.]
	No, this is not the case. Regardless of the actions that Bruno made, with or without small miracles leading him down different possible paths, he could not have killed his grandfather. The reason is that killing his grandfather would create a logical inconsistency; the details of which are laid out in the previous problem. Even if Bruno did not lose his nerve, since Bruno exists some event would take place that would make the murder unsuccessful. 

\end{enumerate}

\end{document}