%%%%%%%%%%%%%%%%%%%%%%%%%%%%%%%%%%%%%%%%%
% University/School Laboratory Report
% LaTeX Template
% Version 3.0 (4/2/13)
%
% This template has been downloaded from:
% http://www.LaTeXTemplates.com
%
% Original author:
% Linux and Unix Users Group at Virginia Tech Wiki 
% (https://vtluug.org/wiki/Example_LaTeX_chem_lab_report)
%
% License:
% CC BY-NC-SA 3.0 (http://creativecommons.org/licenses/by-nc-sa/3.0/)
%
%%%%%%%%%%%%%%%%%%%%%%%%%%%%%%%%%%%%%%%%%

%----------------------------------------------------------------------------------------
%	PACKAGES AND DOCUMENT CONFIGURATIONS
%----------------------------------------------------------------------------------------

\documentclass{article}

\usepackage[version=3]{mhchem} % Package for chemical equation typesetting
\usepackage{siunitx} % Provides the \SI{}{} command for typesetting SI units

\usepackage{graphicx}
\usepackage{caption}
\usepackage{subcaption}

\usepackage{float}

\usepackage[T1]{fontenc} % allow small bold caps

\setlength\parindent{0pt} % Removes all indentation from paragraphs

\renewcommand{\labelenumi}{\alph{enumi}.} % Make numbering in the enumerate environment by letter rather than number (e.g. section 6)

\usepackage[margin=1in]{geometry}

\usepackage{amssymb}

%\usepackage{times} % Uncomment to use the Times New Roman font

%----------------------------------------------------------------------------------------
%	Title
%----------------------------------------------------------------------------------------

\begin{document}
\pagenumbering{gobble}

\title{24.118: Paradox and Infinity}
\author{
  Ryan Lacey <rlacey@mit.edu>\\
  \footnotesize \texttt{Collaborator(s): Evan Thomas}
}
        
\maketitle
        


\begin{enumerate}
\item[1.]
	PLACEHOLDER

\item[2.]
	\begin{enumerate}
	\item[(a)]
		Proof by contradiction, assume that there is some $\alpha$ that is a smaller ordinal than $\emptyset$. Then by our definition $\alpha$ must be an element of $\emptyset$. But $\emptyset$ has no elements. Thus we arrive at a contradiction and can conclude that there is no smaller ordinal than $\emptyset$.
	\item[(b)]
		Proof by contradiction, assume that there is some $\alpha$ containing infinitely many members that is a smaller ordinal than $\omega$. Then $\omega$ must contain some element $N$ that is not within $\alpha$ such that $\alpha$ could be represented as $1_0, 2_0, 3_0, ...$ and that $\omega$ can be represented as $1_0, 2_0, 3_0, ..., N$. By definition, however, $\alpha$ is infinite so any $N$ chosen will be a member of that ordinal. Thus $\alpha$ and $\omega$ contain the same elements, so we can conclude that no infinite sequence can be smaller than $\omega$.
	\end{enumerate}

\bigskip

\item[3.]
	\begin{enumerate}
	\item[(a)]
		Map element $0_0$ from ordinal $\omega$ to the element $\omega$ from the ordinal $\omega'$. Then map each element $n_0$ from ordinal $\omega$ to the element $(n-1)_0$ from the ordinal $\omega'$. Pairing elements in this manner would give the following 1-to-1 correspondence between $\omega$ and $\omega'$\\
		
		\begin{alignat*}{5}
		    0_0  \qquad && 1_0 \qquad && 2_0  \qquad && 3_0  \qquad && ...\\
		    |  \qquad && | \qquad && |  \qquad && | \qquad && \\		    
		    \omega \qquad && 0_0 \qquad && 1_0 \qquad && 1_0  \qquad &&  ...
		\end{alignat*}
		
		\item[(b)]
			The shape of $\omega$ is $|||...$ while the shape of $\omega'$ is $|||...|$ so they are not isomorphic. As was shown with the 1-to-1 correspondence above, the positions had to be rearranged  to get from one ordering to the other (specifically element $\omega$ from $\omega'$ was brought from the last position to the first position). 
	\end{enumerate}

\bigskip

\item[4.]
	\begin{itemize}
	\item[(i)] True
	\item[(ii)] False
	\item[(iii)] True
	\item[(iv)] False
	\item[(v)] True
	\item[(vi)] False
	\item[(vii)] True
	\item[(viii)] False
	\item[(ix)] False
	\item[(x)] True
	\end{itemize}

\end{enumerate}
\end{document}